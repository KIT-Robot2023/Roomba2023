\documentclass[a4paper]{jarticle}
\usepackage{mlstyle2015,eqnarray,graphicx,amssymb,amsmath,latexsym}
\usepackage{stfloats,color,ascmac}
\usepackage{listings,jlisting}
\usepackage{here,bm,url}

\graphicspath{{ImageFile/}}
\title{制御工学特論 中間課題}
\author{Group 8 : 1M1-09 五十嵐 壮英 1M1-45 中西 涼}
\date{\todayAD}

% -- プログラムを記述するための設定 -- %
\lstset{
  basicstyle={\ttfamily},
  identifierstyle={\small},
  commentstyle={\smallitshape},
  keywordstyle={\small\bfseries},
  ndkeywordstyle={\small},
  stringstyle={\small\ttfamily},
  frame={tb},
  breaklines=true,
  columns=[l]{fullflexible},
  numbers=left,
  xrightmargin=0zw,
  xleftmargin=3zw,
  numberstyle={\scriptsize},
  stepnumber=1,
  numbersep=1zw,
  lineskip=-0.5ex
}

% -- 本文 -- %
\begin{document}
\maketitle
% ===================================================================== %
\section{はじめに}
\label{はじめに}
本稿では,ルンバに搭載されているエンコーダに対して設計したオドメトリの性能評価
実験について記述する.まず,第1章では,設計したオドメトリの計算式についてまとめる.
つぎに,第2章では,実験方法について記述する.さらに,第3章では,実験結果について
記述する.最後に,第4章では,本稿の記載内容についてまとめる.
\vspace{2.5mm}
%
% \subsection{aaa-1}
% \label{aaa-1}
% ここに記述する
\vspace{7.5mm}
% ===================================================================== %
\section{bbb}
\label{bbb}
ここに記述する
\vspace{2.5mm}
%
\subsection{bbb-1}
\label{bbb-1}
ここに記述する
\vspace{7.5mm}
% ===================================================================== %
\section{ccc}
\label{ccc}
ここに記述する
\vspace{2.5mm}
%
\subsection{ccc-1}
\label{ccc-1}
ここに記述する
\vspace{7.5mm}
% ===================================================================== %
\section{ddd}
\label{ddd}
プログラムを記述する.
行番号を途中から始めたい場合は,引数として"firstnumber="を設定する
%
\begin{lstlisting}[caption={firstnumber=101の場合}, firstnumber=101]
aaa
\end{lstlisting}
%
\begin{lstlisting}[caption={firstnumber=50の場合}, firstnumber=50]
aaa
\end{lstlisting}
% 
% ===================================================================== %
\end{document}


% == 画像挿入例 1枚 ==
% 
% \begin{figure}[H]
%     \centering
%     % \includegraphics[width=100mm]{~~~.eps}
%     \caption{aaa}
%     \label{aaa}
% \end{figure}
% 
% == 画像挿入例 複数枚 ==
% 
% \begin{figure}[H]
%     \begin{minipage}[H]{0.45\linewidth}
%         \centering
%         \includegraphics[width=50mm]{~~~.eps}
%         \vspace{-2mm}
%         \caption{aaa}
%         \label{aaa}
%     \end{minipage}
%     % 
%     \begin{minipage}[H]{0.45\linewidth}
%         \centering
%         \includegraphics[width=50mm]{~~~.eps}
%         \vspace{-2mm}
%         \caption{aaa}
%         \label{aaa}
%     \end{minipage}
% \end{figure}
% 
% == 表挿入例 ==
% 
% \begin{table}[H]
%     \centering
%     \caption{aaa}
%     \label{aaa}
%     \vspace{1.5mm}
%     \begingroup
%     \renewcommand{\arraystretch}{1.3} % 数値で高さ倍率が変わる
%     \begin{tabular}{|c|c|} 
%         \hline 
%         a & b \\
%         \hline
%         \hline
%         1 & 1 \\
%         \hline 
%     \end{tabular}
%     \endgroup
% \end{table}